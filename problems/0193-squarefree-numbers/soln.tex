\documentclass{article}
\usepackage{amsthm}
\usepackage{amsmath}

\newtheorem{theorem}{Theorem}
\newtheorem{corollary}{Corollary}
\newtheorem{lemma}{Lemma}

\begin{document}

A number $n$ is squarefree if and only if $\lvert \mu(n)\rvert=1$, where $\mu$ is the Mobius function. Computing the Mobius function in linear time (using a linear sieve) would yield an $\mathcal{O}(n)$, which is too slow. 

Instead, consider enumerating the square divisor of a number. Using principle of inclusion-exclusion, we can subtract squarefree numbers that have an odd number of factors, and add squarefree numbers that have an even number of factors. In other words, the answer is $$\sum_{i=1}^{\sqrt{n}}\mu(i)\cdot \left\lfloor\frac{n}{i^2}\right\rfloor.$$

\end{document}
\documentclass{article}
\usepackage{amsthm}
\usepackage{amsmath}

\newtheorem{theorem}{Theorem}
\newtheorem{corollary}{Corollary}
\newtheorem{lemma}{Lemma}

\begin{document}

WLOG, assume that $a\geq b>n$. Then, define positive integers $\alpha=n-a$ and $\beta=n-b$. We have
\begin{align*}
	\frac 1 n&=\frac 1 a+\frac 1 b\\
	\frac 1 n&=\frac{1}{n-\alpha}+\frac{1}{n-\beta}\\
	(n-\alpha)(n-\beta)&=n(n-\beta)+n(n-\alpha)\\
	n^2-(\alpha+\beta)n+\alpha\beta&=n^2-\beta n+n^2-\alpha n\\
	\alpha\beta&=n^2\\
\end{align*}

Thus, the number of solutions for $n$ is the number of divisors of $n^2$. This can be computed efficiently using the \texttt{least_factor} sequence and a simple DP. 

\end{document}

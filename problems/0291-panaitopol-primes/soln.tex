\documentclass{article}
\usepackage{amsthm}
\usepackage{amsmath}

\newtheorem{theorem}{Theorem}
\newtheorem{corollary}{Corollary}
\newtheorem{lemma}{Lemma}

\begin{document}

\begin{lemma}
Every Panaitopol Prime $p$ can be represented as $n^2+(n+1)^2$ for some positive integers $n$. 
\end{lemma}

\begin{proof}

Suppose $p = \frac{x^4-y^4}{x^3+y^3}$. Expanding and simplifying, we get that $p(x^2-xy+y^2) = (x-y)(x^2+y^2)$. Because $p$ is prime, it must be a factor of either $(x - y)$ or $(x^2 + y^2)$. If $p\backslash x-y$, $x^2+y^2$ must divide $x^2-xy+y^2$, which is not possible since $xy$ is positive. Thus, $p\backslash x^2+y^2$ and $x-y\backslash x^2-xy+y^2$.

Because $x-y\backslash x^2-xy$ and $x-y\backslash y^2-xy$, we know that $x-y$ divides $x^2$, $y^2$, and $xy$. Let $a=x^2/(x-y)$, $b=y^2/(x-y)$, and $c=xy/(x-y)$. From these definitions we know that $ab=c^2$ and $a+b-2c=x-y$. 

Let $d=\gcd(a,b,c)$ and $a_0$, $b_0$, and $c_0$ equal $a/d$, $b/d$, and $c/d$. Plugging in for the initial expression for $p$, we get that 

\begin{align*}
p&=\frac{(a+b-2c)(a+b)}{a+b-c} \\
&=\frac{(a_0+b_0-2c_0)(a_0+b_0)d}{a_0+b_0-c_0}.
\end{align*}

Suppose there was some prime $q$ that divided $a_0+b_0-2c_0$ and $a_0+b_0-c_0$. Then, $q\backslash c$, and because $xy=c(x-y)$ we know that either $x$ or $y$ must have a factor of $q$. Without loss of generality, assume this is $x$. Because $x-y\backslash x^2$, we know that $y$ must also have a factor of $q$. It follows that $a_0$ and $b_0$ must consequently be divisible by $q$, violating the maximality of $d$. Thus, $\gcd(a_0+b_0-2c_0, a_0+b_0-c_0) = 1$. 

Thus $a_0+b_0-c_0\backslash d(a_0+b_0)$, and furthermore this quotient must be greater than $1$ because $c_0$ is positive. As $p$ is prime, we know that $a_0+b_0-2c_0 = 1$. By definition, $c_0 = \sqrt{a_0b_0}$, so we get $\sqrt{a_0}-\sqrt{b_0}=1$. Clearly, the only integer solutions are $a_0=(n+1)^2$ and $b_0=n^2$. All that remains is to solve for $x$ and $y$ and plug back into the initial formula for $p$. 

\end{proof}

It suffices to check every $n$ from $1$ to $\sqrt{5\cdot 10^5}$. 


\end{document}

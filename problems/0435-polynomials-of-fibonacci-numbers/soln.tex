\documentclass{article}
\usepackage{amsthm}
\usepackage{amsmath}

\newtheorem{theorem}{Theorem}
\newtheorem{corollary}{Corollary}
\newtheorem{lemma}{Lemma}

\begin{document}

We can define the $i$-th Fibonacci matrix as $$F_n=\begin{bmatrix}1&1\\0&1\end{bmatrix}^n\begin{bmatrix}f_1\\f_0\end{bmatrix}.$$We can get the value of the $x^n$ term by noting that $$f_nx^n = x\cdot f_{n-1}x^{n-1}+x^2\cdot f_{n-2}x^{n-2}.$$ Note that this recurrence is linear, since $x$ can be treated as a constant. We can express express this linear recurrence using matricies, which can be evaluated the following expression in $\mathcal{O}(\log n)$ using matrix exponentiation:$$\begin{bmatrix}x&x^2\\1&0\end{bmatrix}^n\begin{bmatrix}f_1\\f_0\end{bmatrix}.$$To find the sums over these terms, we simply need to add another term to the transition matrix, which yields $$\begin{bmatrix}x&x^2&0\\1&0&0\\x^2&x&1\end{bmatrix}^n\begin{bmatrix}f_1\\f_0\\f_0+f_1\end{bmatrix}.$$

\end{document}